
\subsection{研究意义}
%%%%%%%%
%\newpage
\vspace{-5pt}

普通引用\cite{test};上标引用\citess{test};多篇文章\citess{test,test2,test3}。

有注音的英文:\cite{test}。

参考期刊\cite{test};
参考图书\cite{test2};
参考会议\cite{test5};
参考链接\cite{test4};
参考文件\cite{test6}。

在表格内的第一行设置\verb|\zhkai\ensong\selectfont|,来选择字体,其中\verb|\zhkai\zhsong\enkai\ensong|可以根据需要选择。
\begin{table}[htbp]
	\zhkai\ensong\selectfont%设置表格字体
	\centering  % 显示位置为中间
	\caption{表格}  % 表格标题
	\label{table1}  % 用于索引表格的标签
	%字母的个数对应列数,|代表分割线
	% l代表左对齐,c代表居中,r代表右对齐
	\begin{tabular}{|c|c|c|c|}  
		\hline  % 表格的横线
		& & & \\[-6pt]  %可以避免文字偏上来调整文字与上边界的距离
		第一列&第二列&第三列&第四列 \\  % 表格中的内容,用&分开,\\表示下一行
		\hline
		& & & \\[-6pt]  %可以避免文字偏上 
		0.1&0.2&0.3&0.4 \\
		\hline
	\end{tabular}
\end{table}

对于中文参考文献,bib条目中需要有language = {zh},参见\cite{test2}。

\begin{REF}
\subsection*{参考文献}
\vspace{-50pt}
\bibliographystyle{gbt7714-nsfc}
\bibliography{ref}%参考文献
\end{REF}

